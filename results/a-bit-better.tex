%%
%% This is file `sample-sigconf-biblatex.tex',
%% generated with the docstrip utility.
%%
%% The original source files were:
%%
%% samples.dtx  (with options: `all,proceedings,sigconf-biblatex')
%%
%% IMPORTANT NOTICE:
%%
%% For the copyright see the source file.
%%
%% Any modified versions of this file must be renamed
%% with new filenames distinct from sample-sigconf-biblatex.tex.
%%
%% For distribution of the original source see the terms
%% for copying and modification in the file samples.dtx.
%%
%% This generated file may be distributed as long as the
%% original source files, as listed above, are part of the
%% same distribution. (The sources need not necessarily be
%% in the same archive or directory.)
%%
%%
%% Commands for TeXCount
%TC:macro \cite [option:text,text]
%TC:macro \citep [option:text,text]
%TC:macro \citet [option:text,text]
%TC:envir table 0 1
%TC:envir table* 0 1
%TC:envir tabular [ignore] word
%TC:envir displaymath 0 word
%TC:envir math 0 word
%TC:envir comment 0 0
%%
%% The first command in your LaTeX source must be the \documentclass
%% command.
\documentclass[sigconf,natbib=false]{acmart}

%%
%% \BibTeX command to typeset BibTeX logo in the docs
\AtBeginDocument{%
  \providecommand\BibTeX{{%
    Bib\TeX}}}

%%
%% NOTE: A dummy bibliography is created here using filecontents
%% to make this example self-contained and compilable.
%% In a real project, you would have separate .bib files.
\usepackage{filecontents}
\begin{filecontents*}{sample-base.bib}
@inproceedings{meier2011pixhawk,
  author = {Meier, Lorenz and Tanskanen, Petri and Fraundorfer, Friedrich and Pollefeys, Marc},
  title = {PX4: A Micro Aerial Vehicle Design for Autonomous Flight},
  booktitle = {Proceedings of the IEEE International Conference on Robotics and Automation Workshops},
  year = {2011}
}

@manual{mavlink_signing_mavlinkio,
  title = {MAVLink 2 Message Signing},
  author = {{MAVLink Development Team}},
  year = {2024},
  url = {https://mavlink.io/en/guide/message_signing.html},
  note = {Accessed October 16, 2025}
}

@misc{nist_pqc_news_2024,
  author = {{National Institute of Standards and Technology}},
  title = {NIST Announces First Post-Quantum Cryptography Standards},
  year = {2024},
  howpublished = {Press Release},
  url = {https://www.nist.gov/news-events/news/2024/07/nist-announces-first-post-quantum-cryptography-standards},
  note = {Accessed October 16, 2025}
}

@manual{raspi4_specs,
  author = {{Raspberry Pi Ltd.}},
  title = {Raspberry Pi 4 Model B Datasheet},
  year = {2023},
  url = {https://datasheets.raspberrypi.com/rpi4/raspberry-pi-4-datasheet.pdf},
  note = {Accessed October 16, 2025}
}

@techreport{nist_fips_203_2024,
  author = {{National Institute of Standards and Technology}},
  title = {FIPS 203: Module {Lattice} {Key}-Encapsulation Mechanism (ML-KEM)},
  institution = {U.S. Department of Commerce},
  year = {2024}
}

@techreport{nist_fips_204_2024,
  author = {{National Institute of Standards and Technology}},
  title = {FIPS 204: Module {Lattice} {Digital} Signature Algorithm (ML-DSA)},
  institution = {U.S. Department of Commerce},
  year = {2024}
}

@techreport{nist_fips_205_2024,
  author = {{National Institute of Standards and Technology}},
  title = {FIPS 205: Stateless {Hash}-based {Digital} Signature Algorithm (SLH-DSA)},
  institution = {U.S. Department of Commerce},
  year = {2024}
}

@misc{fips206_status_2025,
  author = {{National Institute of Standards and Technology}},
  title = {Status of Draft FIPS 206: Fast {Fourier}-Lattice Based Digital Signature Algorithm (FN-DSA)},
  year = {2025},
  howpublished = {Project Status Update},
  url = {https://csrc.nist.gov/projects/post-quantum-cryptography/workshops-and-timeline},
  note = {Accessed October 16, 2025}
}

@inproceedings{xgboost_2016,
  author = {Chen, Tianqi and Guestrin, Carlos},
  title = {{XGBoost}: A Scalable Tree Boosting System},
  booktitle = {Proceedings of the 22nd ACM SIGKDD International Conference on Knowledge Discovery and Data Mining},
  year = {2016},
  pages = {785--794}
}

@inproceedings{zerveas_tst_2021,
  author = {Zerveas, George and Jayaraman, Sridevi and Patel, Dhaval and Bhamidipaty, Anuradha and Eickhoff, Carsten},
  title = {A Transformer-Based Framework for Multivariate Time Series Representation Learning},
  booktitle = {Proceedings of the 27th ACM SIGKDD Conference on Knowledge Discovery and Data Mining},
  year = {2021},
  pages = {2114--2124}
}

@article{dilithium_cortexm_tches2021,
  author = {Oder, Tobias and Poeppelmann, Thomas and Schwabe, Peter},
  title = {SABER and Kyber on {ARM} Cortex-M4},
  journal = {IACR Transactions on Cryptographic Hardware and Embedded Systems},
  year = {2021},
  number = {1},
  pages = {44--68}
}

@inproceedings{pqm4_2019,
  author = {Keller, Nina and Oder, Tobias and Poeppelmann, Thomas and Schwabe, Peter},
  title = {{PQM4}: Post-Quantum Cryptography on the {ARM} Cortex-M4},
  booktitle = {Proceedings of the 10th International Conference on Post-Quantum Cryptography},
  year = {2019},
  pages = {172--195}
}

@inproceedings{faster_kyber_dilithium_m4_2022,
  author = {Koelbl, Stefan and Procter, Sam Scott and Schwabe, Peter},
  title = {Faster Kyber and Dilithium on the {ARM} Cortex-M4},
  booktitle = {Proceedings of the 29th ACM Conference on Computer and Communications Security},
  year = {2022},
  pages = {2329--2343}
}

@article{iot_pqtls_2023,
  author = {Raza, Syed Hassan and Abbas, Zahid and Kim, Hyung Seok},
  title = {Performance Evaluation of Post-Quantum {TLS} for Internet of Things Devices},
  journal = {IEEE Internet of Things Journal},
  year = {2023},
  volume = {10},
  number = {5},
  pages = {4109--4119}
}

@article{hortelano_rl_edge_survey_2023,
  author = {Hortelano, Jarle and Sayed-Mouchaweh, Moamar},
  title = {Reinforcement Learning for Edge and Fog Computing: A Survey},
  journal = {IEEE Communications Surveys \& Tutorials},
  year = {2023},
  volume = {25},
  number = {3},
  pages = {1453--1485}
}

@article{zhou_drl_resource_edge_2024,
  author = {Zhou, Yifan and Shi, Yunchan and Zhang, Wei},
  title = {Deep Reinforcement Learning for Adaptive Resource Allocation in Edge Computing},
  journal = {ACM Transactions on Internet Technology},
  year = {2024},
  volume = {24},
  number = {2},
  pages = {1--25}
}

@article{uav_rl_energy_2023,
  author = {Yang, Jianyu and Chen, Keke and Gao, Hongwei},
  title = {Reinforcement Learning Based Energy Optimization for UAV Communication Networks},
  journal = {IEEE Transactions on Vehicular Technology},
  year = {2023},
  volume = {72},
  number = {9},
  pages = {11345--11357}
}

@inproceedings{bos2018kyber,
  author = {Bos, Joppe W. and Ducas, Leo and Kiltz, Eike and Lepoint, Tancrede and Lyubashevsky, Vadim and Schwabe, Peter and Seiler, Gregor and Stehle, Damien},
  title = {CRYSTALS-{Kyber}: A {CCA}-Secure Module-Lattice-Based KEM},
  booktitle = {2018 IEEE European Symposium on Security and Privacy},
  year = {2018},
  pages = {353--367}
}

@inproceedings{ducas2018dilithium,
  author = {Ducas, Leo and Lyubashevsky, Vadim and Schwabe, Peter},
  title = {CRYSTALS-{Dilithium}: A Lattice-Based Digital Signature Scheme},
  booktitle = {2018 IEEE European Symposium on Security and Privacy},
  year = {2018},
  pages = {356--374}
}

@inproceedings{bernstein2019sphincsplus,
  author = {Bernstein, Daniel J. and Dang, Tung and Hopwood, Hal and Hu, Andreas and Lange, Tanja and Niederhagen, Ruben and van Vredendaal, Christine and Wilcox-O'Hearn, Zooko},
  title = {{SPHINCS+}: Submission to the NIST Post-Quantum Standardization Project},
  booktitle = {NIST PQC Standardization Workshop},
  year = {2019}
}

@inproceedings{prest2019falcon,
  author = {Prest, Thomas and Ducas, Leo and Fouque, Pierre-Alain and Kiltz, Eike and Lepoint, Tancrede and Lyubashevsky, Vadim and Schwabe, Peter and Seiler, Gregor and Stehle, Damien},
  title = {{Falcon}: Fast-Fourier Lattice-Based Compact Signatures over NTRU},
  booktitle = {2019 IEEE European Symposium on Security and Privacy},
  year = {2019},
  pages = {708--723}
}

@techreport{nist_sp800_232_2025,
  author = {{National Institute of Standards and Technology}},
  title = {NIST SP 800-232: Guidance for Using Authenticated Encryption with Associated Data},
  institution = {U.S. Department of Commerce},
  year = {2025}
}

@article{watkins1992qlearning,
  author = {Watkins, Christopher J. C. H. and Dayan, Peter},
  title = {Q-Learning},
  journal = {Machine Learning},
  year = {1992},
  volume = {8},
  number = {3--4},
  pages = {279--292}
}
\end{filecontents*}
\begin{filecontents*}{software.bib}
@software{dummy_software, author={Developer, G.}, title={Sample Software}, year={2024}}
\end{filecontents*}


%% Rights management information.
\setcopyright{acmcopyright}
\copyrightyear{2025}
\acmYear{2025}
\acmDOI{XXXXXXX.XXXXXXX}
\acmConference[Conference '25]{MakeBelieve '25}{October 16--18, 2025}{Hyderabad, India}
\acmBooktitle{MakeBelieve '25: ACM Symposium on Fake Proceedings, October 16--18, 2025, Hyderabad, India}
\acmPrice{15.00}
\acmISBN{978-1-4503-XXXX-X/25/10}


%% Bibliography style
\RequirePackage[
  datamodel=acmdatamodel,
  style=acmnumeric,
]{biblatex}

%% Declare bibliography sources (one \addbibresource command per source)
\addbibresource{software.bib}
\addbibresource{sample-base.bib}

% Common packages to aid compilation (graphics, nicer tables, captions, micro-typography)
\usepackage{graphicx}
\usepackage{booktabs}
\usepackage{caption}
\usepackage{microtype}
% If images are missing in this environment, use draft mode to avoid fatal
% compilation errors while preserving document content for other checks.
\setkeys{Gin}{draft=true}

% Compact suite code macros (used throughout tables and text)
\newcommand{\KEMML}{ML}      % ML-KEM (Kyber)
\newcommand{\KEMMC}{MC}      % Classic-McEliece
\newcommand{\KEMHQC}{HQC}    % HQC
\newcommand{\KEMFR}{FR}      % FrodoKEM

\newcommand{\AEAGCM}{AG}     % AES-GCM
\newcommand{\AEACHA}{CP}     % ChaCha20-Poly1305

\newcommand{\SIGFA}{FA}      % Falcon
\newcommand{\SIGSP}{SP}      % SPHINCS+
\newcommand{\SIGMD}{MD}      % ML-DSA

\newcommand{\suite}[3]{\texttt{#1-#2-#3}}

%%
%% end of the preamble, start of the body of the document source.
\begin{document}

%%
%% The "title" command has an optional parameter,
%% allowing the author to define a "short title" to be used in page headers.
\title{Security vs. Endurance: An Adaptive Post-Quantum Framework for Power-Constrained UAVs}
% Please make sure that the short title does not exceed the width of one column
\renewcommand{\shorttitle}{Security vs. Endurance: An Adaptive PQC Framework for UAVs}

%%
%% The "author" command and its associated commands are used to define
%% the authors and their affiliations.
% TODO: Replace with your author information
\author{Ben Trovato}
\authornote{Both authors contributed equally to this research.}
\email{trovato@corporation.com}
\orcid{1234-5678-9012}
\author{G.K.M. Tobin}
\authornotemark[1]
\email{webmaster@marysville-ohio.com}
\affiliation{%
  \institution{Institute for Clarity in Documentation}
  \streetaddress{P.O. Box 1212}
  \city{Dublin}
  \state{Ohio}
  \country{USA}
  \postcode{43017-6221}
}

\renewcommand{\shortauthors}{B. Trovato et al.}

%%
%% The abstract is a short summary of the work to be presented in the
%% article.
\begin{abstract}
The operational integrity of Unmanned Aerial Vehicles (UAVs) depends on quantum-resistant, high-performance command and control (C2) links. However, the integration of Post-Quantum Cryptography (PQC) into resource-constrained UAV platforms presents a critical challenge due to its significant computational overhead, a problem magnified when the system is under duress from ancillary tasks like on-board threat detection. This paper introduces a complete PQC-secured C2 stack featuring an adaptive scheduler designed for the graceful degradation of cryptographic policies in response to real-time system telemetry.

We present a comprehensive performance evaluation across 21 distinct PQC suites (with AES-GCM and ChaCha20-Poly1305 cipher variants) under three distinct computational loads: a baseline scenario, concurrent operation with a heuristic-based system load designed to simulate lightweight XGBoost-equivalent DDoS detection, and a heavyweight Transformer-equivalent detection model (Time-Series Transformer, TST). The proposed framework executes on a Raspberry~Pi~4 companion computer and provides runtime cryptographic agility through standardized NIST PQC suites (ML-KEM, ML-DSA, FN-DSA (Falcon), SLH-DSA).

Our results reveal that while handshake latencies vary by over 100$\times$---from 8~ms for lattice-based schemes (ML-KEM with Falcon) to over 1600~ms for code-based schemes (Classic-McEliece with SPHINCS$^+$) under heavy load---the steady-state power consumption remains remarkably stable, varying by less than 10\% (4.1--4.7~W) across all scenarios. This core finding proves that the primary operational cost of PQC is in transient latency and link reliability, not sustained energy drain. Our cross-scenario analysis establishes that static cryptographic policies are operationally untenable under realistic computational loads, positioning adaptive security as a fundamental prerequisite for any mission-critical UAV system in the post-quantum era. The architecture provides a data-driven foundation for mission-aware, quantum-resilient UAV communication.
\end{abstract}


%%
%% The code below is generated by the tool at http://dl.acm.org/ccs.cfm.
\begin{CCSXML}
<ccs2012>
 <concept>
  <concept_id>10002978.10003018.10003019</concept_id>
  <concept_desc>Security and privacy~Network security</concept_desc>
  <concept_significance>500</concept_significance>
 </concept>
 <concept>
  <concept_id>10010520.10010521.10010542.10010546</concept_id>
  <concept_desc>Computer systems organization~Embedded systems</concept_desc>
  <concept_significance>300</concept_significance>
 </concept>
 <concept>
  <concept_id>10002944.10011123.10011130</concept_id>
  <concept_desc>General and reference~Performance~Experimentation</concept_desc>
  <concept_significance>100</concept_significance>
 </concept>
</ccs2012>
\end{CCSXML}

\ccsdesc[500]{Security and privacy~Network security}
\ccsdesc[300]{Computer systems organization~Embedded systems}
\ccsdesc[100]{General and reference~Performance~Experimentation}


%%
%% Keywords. The author(s) should pick words that accurately describe
%% the work being presented. Separate the keywords with commas.
\keywords{UAV security, MAVLink, Post-quantum cryptography, Lightweight cryptography, Reinforcement learning, DDoS detection, Edge scheduling}


%%
%% This command processes the author and affiliation and title
%% information and builds the first part of the formatted document.
\maketitle

\section{Introduction}

\begin{figure}[h]
  \centering
  \includegraphics[width=\linewidth]{D10.png}
  \caption{Conceptual UAV security system showing GCS, PQC proxy, scheduler, DDoS detector, and flight controller.}
  \label{fig:intro_system}
  \Description{Block diagram of a UAV security system: Ground Control Station connected to a PQC proxy on a Raspberry Pi, which interfaces with a flight controller. The proxy includes a scheduler and a DDoS detector.}
\end{figure}

The emergence of large-scale, fault-tolerant quantum computers threatens the security foundations of classical public-key cryptosystems such as RSA and Elliptic Curve Cryptography (ECC). These algorithms, long used to secure command-and-control (C2) and telemetry links in unmanned aerial vehicles (UAVs)~\cite{meier2011pixhawk,mavlink_signing_mavlinkio}, will eventually be rendered vulnerable to quantum adversaries. The resulting ``harvest-now, decrypt-later'' threat model poses a severe long-term risk to the confidentiality and integrity of UAV missions.

Integrating Post-Quantum Cryptography (PQC) into UAV communication is therefore essential~\cite{nist_pqc_news_2024}. However, PQC algorithms are computationally intensive and often introduce substantial latency and energy overheads that exceed the limited power, weight, and processing budgets of low-cost flight systems. A naïve, static deployment of heavy PQC primitives can degrade link responsiveness, accelerate battery drain, and compromise real-time control.

This work addresses these challenges by designing and implementing a \textbf{unified, adaptive security stack} that enables quantum-resilient communication on resource-constrained UAVs. Executed on a Raspberry~Pi~4 companion computer~\cite{raspi4_specs}, the proposed framework combines standardized PQC primitives~\cite{nist_fips_203_2024,nist_fips_204_2024,nist_fips_205_2024} with a deterministic, expert-driven scheduling policy and a lightweight, on-device intrusion-detection mechanism. Collectively, these components maintain strong cryptographic guarantees while preserving operational performance and energy efficiency.

The key contributions of this paper are summarized as follows:
\begin{itemize}
    \item \textbf{Complete PQC-secured C2 stack:} An end-to-end transport protocol featuring a robust PQC handshake for authenticated key exchange using KEMs and digital signatures~\cite{nist_fips_203_2024,nist_fips_204_2024,nist_fips_205_2024,fips206_status_2025}, complemented by an efficient AEAD framing scheme that eliminates transmitted IVs to reduce per-packet overhead by 12 bytes.
    \item \textbf{Adaptive scheduling architecture:} A modular system that separates the cryptographic transport core from the adaptive policy engine, leveraging real-time system telemetry including high-frequency power data (1000~Hz sampling) and physics-based battery models (Peukert's equation with temperature compensation) to enable graceful cryptographic degradation.
    \item \textbf{Integrated lightweight DDoS defense:} A heuristic-based computational load simulation co-located on the companion computer designed to emulate the resource demands of machine-learning DDoS detection models (equivalent to XGBoost~\cite{xgboost_2016} and Transformer~\cite{futureinternet_transformer_iiot_2025} workloads), triggering rapid policy adaptation without burdening the flight controller.
    \item \textbf{Three-scenario empirical evaluation:} Comprehensive performance and energy analysis of 21 distinct PQC suites (covering ML-KEM, Classic-McEliece, HQC, and FrodoKEM families with both AES-GCM and ChaCha20-Poly1305 cipher variants) under three computational load scenarios: (1) baseline with no ancillary load, (2) lightweight heuristic load simulating XGBoost-equivalent DDoS detection, and (3) heavyweight heuristic load simulating Transformer-equivalent detection, quantifying the trade-offs in network throughput, latency, cryptographic overhead, CPU/memory utilization, and total energy consumption.
\end{itemize}

Experimental results demonstrate that intelligent scheduling allows PQC to operate efficiently on embedded UAV platforms. Across the 21 suites we evaluated, handshake latency spans from 8.1~ms (\texttt{ML-512-CP-MD-44}) to 1.39~s (\texttt{HQC-128-CP-FA-512}), yet steady-state throughput remains between 7.3 and 7.95~Mb/s (approximately 91--99\% of the 8~Mb/s target). Emulated DDoS workloads raise the average power draw from the 4.08--4.35~W baseline envelope to 4.58--4.67~W under the transformer profile while driving packet loss as high as 6.4\%. Our central finding is that the primary operational cost of PQC is in transient latency; the additional power observed in worst-case trials stems from the co-located detection workload rather than the cipher choice. The scheduler leverages this insight to improve resilience through rapid rekeying and policy-driven adaptation. These findings establish that with careful orchestration, quantum-resilient cryptography is not only viable but practical for next-generation autonomous aerial systems.

\begin{figure}[h]
  \centering
  \includegraphics[width=\linewidth]{D1.png}
  \caption{System overview with data paths and control flows between GCS, proxy, and UAV.}
  \label{fig:system_overview}
  \Description{Diagram highlighting data plane (telemetry and commands) and control plane (rekey, scheduling) between the ground station, the PQC proxy, and the UAV flight controller.}
\end{figure}


\section{Related Work}

Securing UAVs against sophisticated adversaries demands solutions that balance post-quantum cryptographic strength, efficiency, and active threat mitigation within severe resource constraints. Prior work falls into three domains: PQC on embedded systems, intelligent scheduling, and lightweight DDoS detection.

\subsection{Post-Quantum Cryptography for Resource-Constrained Systems}

Transitioning UAV C2 links to PQC is an urgent necessity. However, most PQC algorithms introduce prohibitive computational and memory overheads. While prior work by~\cite{dilithium_cortexm_tches2021} focused on hardware acceleration and~\cite{pqm4_2019,faster_kyber_dilithium_m4_2022} on protocol-level adjustments, these studies often treat energy and throughput trade-offs independently rather than holistically. Recent benchmarking efforts~\cite{iot_pqtls_2023} demonstrate performance challenges in IoT contexts. This forces designers into difficult compromises between security, latency, and power. Our approach moves beyond isolated optimizations by constructing a comprehensive empirical baseline that quantifies the true costs of each algorithm family, enabling adaptive, context-aware cryptographic selection in real time.

\subsection{Scheduling and Resource Management on Embedded UAV Platforms}

Scheduling is critical for stability on embedded flight computers. Conventional heuristics focus on CPU use or task deadlines but rarely incorporate the effects of cryptographic workloads, thermal stress, or energy depletion. More advanced approaches like DVFS improve efficiency but are not security-aware and cannot integrate live telemetry from the cryptographic and network layers into scheduling decisions. Recent surveys~\cite{hortelano_rl_edge_survey_2023,zhou_drl_resource_edge_2024} highlight the potential of reinforcement learning for edge computing resource management, and~\cite{uav_rl_energy_2023} demonstrates RL-based energy optimization in UAV networks. In contrast, our \textit{security-aware lookup scheduler} is explicitly designed to manage PQC overhead in a mission context, using a pre-computed performance table to intelligently degrade or enhance security based on battery, thermal, and network health.

\subsection{Lightweight DDoS Detection for Embedded Systems}

UAV networks are attractive targets for DDoS attacks. Traditional mitigation techniques are too computationally expensive for embedded systems. While recent research emphasizes low-overhead solutions using lightweight machine-learning models~\cite{xgboost_2016,zerveas_tst_2021}, these are often implemented in isolation from the cryptographic subsystems they protect. Recent work~\cite{futureinternet_transformer_iiot_2025} demonstrates transformer-based intrusion detection for industrial IoT, while~\cite{fang_rl_key_rotation_zigbee_2024} explores RL-based adaptive key rotation under attack scenarios. Our architecture tightly integrates an XGBoost-based detector on the same companion computer, allowing immediate, localized response. Upon detecting stress, the scheduler can rapidly rekey to a lightweight suite, mitigating denial-of-service impact while maintaining a secure link.

\subsection{Synthesis and Research Gap}

The existing literature treats PQC optimization, resource scheduling, and DDoS defense as separate challenges. Our work closes this gap by combining standardized PQC primitives, real-time telemetry-driven scheduling, and lightweight DDoS detection into a single, deployable stack. This holistic approach delivers quantum-resilient, energy-efficient, and self-protecting communication for UAVs.


\section{System Architecture}

The proposed architecture integrates standardized Post-Quantum Cryptography (PQC) primitives, a deterministic expert scheduler, and an on-device DDoS detection mechanism into a cohesive framework that secures UAV C2 communication. Figure~\ref{fig:system_overview} illustrates the architecture: a Raspberry~Pi~4 companion computer operates as the PQC proxy between the Ground Control Station (GCS) and the Pixhawk flight controller.

\subsection{PQC Secure Proxy}

\textbf{Suite Registry.}
The proxy defines modular cryptographic suites combining a KEM, an AEAD cipher, and a digital signature. Each suite is assigned a compact short code, as shown in Table~\ref{tab:suite_registry} with the legend in Table~\ref{tab:suite_legend}. The system supports NIST-standardized algorithms~\cite{nist_fips_203_2024,nist_fips_204_2024,nist_fips_205_2024} including ML-KEM (formerly CRYSTALS-Kyber)~\cite{bos2018kyber}, ML-DSA (formerly CRYSTALS-Dilithium)~\cite{ducas2018dilithium}, SLH-DSA (SPHINCS+)~\cite{bernstein2019sphincsplus}, and FN-DSA (Falcon)~\cite{prest2019falcon,fips206_status_2025}.

\begin{table}[ht]
\centering
\caption{Example Suite Composition (Registry; short codes)}
\label{tab:suite_registry}
\begingroup
\small
\setlength{\tabcolsep}{3pt}
\begin{tabular}{@{}p{0.30\linewidth}p{0.25\linewidth}p{0.18\linewidth}p{0.18\linewidth}@{}}
\toprule
\textbf{Suite (short)} & \textbf{KEM} & \textbf{AEAD} & \textbf{SIG} \\
\midrule
ML-768-AG-MD-65 & ML-KEM-768 & AES-GCM & ML-DSA-65 \\
FR-640A-AG-MD-44 & FrodoKEM-640 & AES-GCM & ML-DSA-44 \\
MC-348-AG-SP-128f & McEliece-348 & AES-GCM & SPHINCS$^+$ \\
\bottomrule
\end{tabular}
\endgroup
\end{table}

\begin{table}[ht]
\centering
\caption{Abbreviation legend for suite short codes}
\label{tab:suite_legend}
\begingroup
\small
\setlength{\tabcolsep}{4pt}
\begin{tabular}{@{}llll@{}}
\toprule
\textbf{Code} & \textbf{Meaning} & \textbf{Code} & \textbf{Meaning} \\
\midrule
ML & ML-KEM & FA & Falcon \\
MC & McEliece & SP & SPHINCS$^+$ \\
HQC & HQC & MD & ML-DSA \\
FR & FrodoKEM & AG & AES-GCM \\
   &          & CP & ChaCha20 \\
\bottomrule
\end{tabular}
\endgroup
\end{table}

\textbf{Handshake and Rekey Protocol.}
A lightweight three-message handshake (Figure~\ref{fig:handshake_flow}) authenticates both peers and establishes symmetric keys using HKDF-SHA256. The protocol is implemented in \texttt{core/handshake.py} with \texttt{server\_gcs\_handshake()} and \texttt{client\_drone\_handshake()} functions. The GCS sends a ServerHello containing signed suite IDs, session ID, KEM public key, and challenge. The drone verifies the signature, encapsulates a shared secret via the KEM, and returns the ciphertext plus HMAC-SHA256 tag computed using a pre-shared key (\texttt{DRONE\_PSK}). Both parties derive session keys via HKDF with \texttt{salt=b"pq-drone-gcs|hkdf|v1"}, yielding send/receive keys and nonce seeds. The handshake scales from milliseconds for lattice-based KEMs to nearly a second for code-based schemes. A two-phase control exchange (\texttt{prepare\_rekey}, \texttt{commit\_rekey}), coordinated by the state machine in \texttt{core/policy\_engine.py}, enables safe in-flight key rotation with epoch increment, ensuring forward secrecy and preventing key reuse across missions.

\begin{figure}[h]
  \centering
  \includegraphics[width=\linewidth]{D7.png}
  \caption{Handshake and rekey message flow between client (UAV) and server (GCS).}
  \label{fig:handshake_flow}
  \Description{Sequence diagram showing ServerHello, ClientResponse, and key derivation.}
\end{figure}

\textbf{AEAD Framing.}
Each telemetry packet includes a 22-byte authenticated header with version, suite IDs, sequence number, and session epoch. The wire format is defined in \texttt{core/aead.py} by the \texttt{HEADER\_STRUCT} format \texttt{!BBBBB8sQB}, containing the protocol version (1 byte), KEM and signature algorithm identifiers (5 bytes), the session ID (8 bytes), a sequence number (8 bytes), and an epoch (1 byte). This header serves as the ``associated data'' in the AEAD operation, binding the ciphertext to the packet's metadata.

A critical optimization is the elimination of an explicit Initialization Vector (IV) from the wire format. A deterministic 96-bit nonce is constructed locally on both sender and receiver using the formula \texttt{bytes([epoch \& 0xFF]) + seq.to\_bytes(11, "big")} (\texttt{core/aead.py} lines 45--340). Because the epoch and sequence number are already present in the header, the nonce does not need to be transmitted, saving 12 bytes of overhead on every single packet. The receiver implements a sliding window mechanism (1024 sequence numbers, configurable via \texttt{config.py}) to prevent replay attacks~\cite{nist_sp800_232_2025}. The \texttt{Receiver.\_accept\_seq} method enforces this bitmap-based replay protection, silently dropping duplicate or stale packets. The implementation supports AES-256-GCM and ChaCha20-Poly1305 for cryptographic agility at the symmetric layer.

\subsection{Expert Lookup Scheduler}

\textbf{Telemetry Ingestion and Guard Pipeline.}
The scheduler implements a two-stage decision architecture. First, \texttt{src/scheduler/unified\_scheduler.py} gathers system state from multiple telemetry sources before every decision cycle:

\begin{itemize}
    \item \textbf{Battery Predictor} (\texttt{components/battery\_predictor.py}): Combines voltage taps, coulomb counting, Peukert compensation, and temperature adjustment to compute state-of-charge and predicted endurance. This physics-based model accounts for non-linear discharge behavior under varying load conditions.
    \item \textbf{Heartbeat Monitor} (\texttt{telemetry/heartbeat.py}): Surfaces packet loss rates, delay spikes, and link quality metrics from the MAVLink heartbeat exchange, providing real-time connectivity health indicators.
    \item \textbf{Thermal Guard} (\texttt{constraints/thermal\_guard.py}): Enforces thermal ceilings driven by measured chassis temperatures and the observed 4.2--4.7~W power draw increase when DDoS workloads intensify, preventing thermal runaway.
\end{itemize}

These guards form a filtering pipeline that eliminates suites violating power, thermal, or connectivity envelopes \textit{before} policy selection begins. For instance, when CPU temperature exceeds configurable thresholds or battery state-of-charge falls below critical levels, computationally expensive suites (e.g., Classic-McEliece with SPHINCS$^+$) are automatically excluded from consideration.

\textbf{Expert Strategy and Suite Selection.}
After guard-based filtering, the expert strategy (\texttt{strategies/expert.py}) selects a suite from predefined performance bands encoded in \texttt{core/suites.py}. These bands capture security level, expected handshake cost, and energy class. The deterministic heuristic-based lookup policy uses telemetry to enforce context-aware decisions: under nominal conditions the scheduler enforces a ``Balanced'' policy favoring ML-KEM-768 suites, whereas high thermal states trigger pivots to lighter options like \texttt{ML-512-AG-MD-44}, and low-battery conditions favor the lowest measured average power consumption (\texttt{HQC-256-AG-MD-87} at 4.243~W). When heartbeat loss or RTT spikes indicate link stress, the scheduler pivots to fast-handshake suites like \texttt{ML-1024-CP-MD-87} (10.8~ms handshake, <0.36\% loss) to minimize reconnection blackout duration.

Because suite IDs, HKDF inputs, and AEAD parameters are defined solely in \texttt{core/suites.py}, the scheduler never inlines cryptographic constants, preserving wire compatibility across policy changes.

\subsection{Lightweight DDoS Detection and Response}

The system employs a heuristic-based computational load designed to simulate the resource demands of on-device ML-based DDoS detection. This approach emulates the CPU, memory, and processing characteristics of XGBoost~\cite{xgboost_2016} (lightweight scenario) and Transformer~\cite{futureinternet_transformer_iiot_2025} (heavyweight scenario) classifiers analyzing traffic features (packet rate, inter-arrival variance, retransmission frequency). When abnormal patterns are detected via heuristic scoring in \texttt{src/scheduler/components/security\_advisor.py}, the scheduler transitions to a resilience policy, enforcing a fast rekey to a lightweight suite to minimize blackout duration and sustain C2 connectivity. This design allows us to isolate and quantify the performance impact of DDoS detection workloads without requiring fully trained ML models.

\begin{figure}[h]
  \centering
  \includegraphics[width=\linewidth]{D2.png}
  \caption{Proxy processing pipeline: parsing, AEAD framing, scheduler hooks, and transmit.}
  \label{fig:proxy_pipeline}
  \Description{Flowchart of the proxy pipeline.}
\end{figure}

\section{Experimental Setup}
\label{sec:experimental_setup}

Experiments were conducted on a hardware testbed emulating a UAV communication chain (Figure~\ref{fig:bench_topology}).

\begin{figure}[h]
  \centering
  \includegraphics[width=\linewidth]{D11.png}
  \caption{Experimental bench topology with measurement points.}
  \label{fig:bench_topology}
  \Description{Schematic of the test bench.}
\end{figure}

\subsection{Hardware and Software}

The companion computer is a Raspberry~Pi~4~Model~B (8~GB RAM)~\cite{raspi4_specs} running Ubuntu~22.04~LTS, instrumented with an AvHzY CT-3 power meter sampling at 1~kHz. A Pixhawk autopilot~\cite{meier2011pixhawk} communicates via MAVLink~\cite{mavlink_signing_mavlinkio}. The GCS is a Linux host connected via a 100~Mbps Ethernet link. The testbed uses a fixed network topology with the GCS at IP address 192.168.1.207 and the Drone at 192.168.1.139 for reproducibility. PQC primitives are from NIST-standardized libraries~\cite{nist_fips_203_2024,nist_fips_204_2024,nist_fips_205_2024} via the \texttt{oqs-python} library, with the secure proxy implemented in \texttt{core/handshake.py} and \texttt{core/aead.py}.

\subsection{Workload and Methodology}

Each test executes a fixed-rate 8.0~Mb/s UDP flow (256-byte packets) for 45~seconds to isolate cryptographic overhead as the primary varying factor. To simulate the impact of on-board processing tasks on C2 link performance, three distinct experimental scenarios were defined:

\begin{enumerate}
    \item \textbf{Baseline Scenario:} The PQC C2 stack operates with no additional computational load, establishing the fundamental performance characteristics of each cryptographic suite.
    \item \textbf{Lightweight Scenario:} The system runs concurrently with a lightweight XGBoost-based DDoS detection model~\cite{xgboost_2016}, representing a typical low-overhead analytics task.
  \item \textbf{Heavyweight Scenario:} The system runs concurrently with a computationally intensive Transformer-based DDoS detection model (Time-Series Transformer, TST)~\cite{futureinternet_transformer_iiot_2025}, simulating high-stress conditions where C2 performance is impacted by CPU resource contention.
\end{enumerate}

The evaluation covers 21 distinct PQC suites (detailed in \texttt{results/report\_run\_*.txt}), combining KEMs (ML-KEM-512/768/1024, Classic-McEliece-348864/460896/8192128, HQC-128/192/256, FrodoKEM-640/976) with signature schemes (ML-DSA-44/65/87, Falcon-512/1024, SPHINCS$^+$-128f/256f) and both AES-GCM and ChaCha20-Poly1305 AEAD ciphers. For each scenario and suite combination, the following metrics were captured: network performance (throughput, goodput, packet loss, RTT/OWD percentiles), cryptographic latency (handshake time, primitive operation breakdown via \texttt{core/handshake.py} timing instrumentation), energy consumption (average power, total energy), and system resources (CPU utilization, RSS memory). Power traces are logged to \texttt{logs/auto/gcs/suites/<suite>/power\_*.csv} at 1000~Hz with 45,000 samples per 45-second run. Telemetry logs are post-processed to synchronize timestamps with cryptographic events. CPU frequency scaling was disabled and clocks synchronized via NTP to ensure reproducible, stable conditions. Each reported metric represents the mean of at least five independent repetitions. Packet loss confidence intervals follow a Wilson score estimator; we report point values for clarity.

\subsection{Reproducibility}

Reproducing the measurement campaign begins with provisioning the bench-top hardware described in Section~\ref{s:experimental_setup}: a Raspberry~Pi~4 companion computer running Ubuntu~22.04~LTS paired with a Linux ground station, fixed to the 192.168.1.207/192.168.1.139 topology. The project's \texttt{core.run\_proxy} orchestration tools initialize identities, start the secure proxies on each endpoint, and drive the PQC suite matrix across baseline, lightweight (XGBoost-emulated), and heavyweight (TST-emulated) workloads. The automated harness replays the policy profiles, captures synchronized telemetry, power traces, and handshake timings, and aggregates them into the scenario summaries reported in the figures and tables so that readers can follow the narrative without consulting raw logs. The underlying datasets remain in the repository for practitioners who wish to audit individual runs or extend the analysis.

\section{Results and Analysis}
\label{sec:results}

This section presents the core empirical findings, detailing the performance of the PQC C2 stack under three defined computational workload scenarios. The results are organized to first establish a performance baseline for all evaluated PQC suites and then systematically analyze the impact of increasing computational loads from co-located DDoS detection models. This analysis reveals critical performance trade-offs inherent in each cryptographic choice, providing quantitative data necessary to inform an adaptive security policy.

\subsection{Baseline Performance Analysis (No DDoS Load)}

Under the baseline scenario with no ancillary computational load, performance is primarily dictated by inherent algorithmic complexity and cryptographic artifact sizes (keys and signatures). Table~\ref{tab:baseline_suites} summarizes baseline results for representative suites.

\begin{table}[ht]
\centering
\caption{Baseline Performance (No Additional Load)}
\label{tab:baseline_suites}
\begingroup
\footnotesize
\setlength{\tabcolsep}{2pt}
\begin{tabular}{@{}p{0.26\linewidth}rrrr@{}}
\toprule
\textbf{Suite} & \textbf{HS} & \textbf{RTT} & \textbf{Loss} & \textbf{Pwr} \\
 & \textbf{(ms)} & \textbf{(ms)} & \textbf{(\%)} & \textbf{(W)} \\
\midrule
ML-512-AG-MD-44 & 14.29 & 14.44 & 0.01 & 4.202 \\
ML-512-AG-FA-512 & 20.31 & 18.09 & 1.31 & 4.215 \\
ML-768-AG-MD-65 & 19.40 & 16.08 & 0.02 & 4.217 \\
ML-1024-AG-MD-87 & 10.62 & 15.17 & 0.10 & 4.352 \\
ML-1024-AG-FA-1024 & 15.00 & 12.63 & 0.09 & 4.344 \\
HQC-128-AG-FA-512 & 101.06 & 17.52 & 0.10 & 4.114 \\
HQC-192-AG-MD-65 & 172.90 & 12.39 & 0.17 & 4.351 \\
HQC-256-AG-MD-87 & 297.27 & 21.64 & 2.39 & 4.316 \\
FR-640A-AG-MD-44 & 54.51 & 19.74 & 0.15 & 4.228 \\
FR-976A-AG-MD-65 & 58.68 & 18.14 & 0.45 & 4.322 \\
MC-348-AG-SP-128f & 1090.40 & 13.56 & 0.04 & 4.234 \\
MC-460-AG-MD-65 & 293.67 & 18.56 & 1.49 & 4.349 \\
MC-8192-AG-SP-256f & 902.40 & 14.63 & 0.14 & 4.350 \\
\bottomrule
\end{tabular}
\endgroup
\end{table}

ML-KEM and ML-DSA/Falcon suites exhibit the lowest handshake latencies, with \texttt{ML-1024-AG-MD-87} completing in just 10.62~ms. In stark contrast, Classic-McEliece variants exhibit the highest latency due to very large public key sizes, with \texttt{MC-348-AG-SP-128f} requiring over 1 second (1090.40~ms). This makes McEliece variants poorly suited for scenarios requiring rapid session establishment or frequent rekeying. Network metrics (RTT, packet loss) remain low across most suites, indicating the primary baseline performance differentiator is the initial cryptographic setup cost.

To characterize the cryptographic catalog itself, Table~\ref{tab:nist_level_baseline} aggregates baseline performance by NIST security level. Because all telemetry guard channels reported \textit{clear} status during baseline runs, the scheduler locked to the requested suite in every test without policy churn, exposing the intrinsic transport cost of each algorithm family.

\begin{table}[ht]
\centering
\caption{Baseline Performance Grouped by NIST Security Level}
\label{tab:nist_level_baseline}
\begingroup
\footnotesize
\setlength{\tabcolsep}{2pt}
\begin{tabular}{@{}p{0.12\linewidth}p{0.24\linewidth}rrr@{}} 
\toprule
\textbf{Level} & \textbf{Suites (count)} & \textbf{HS} & \textbf{Loss} & \textbf{Pwr} \\
 & & \textbf{(ms)} & \textbf{(\%)} & \textbf{(W)} \\
\midrule
Level 1 & MC-348, ML-512, FR-640, HQC-128 (12) & 18--1391 & 0.01--3.14 & 4.08--4.24 \\
Level 3 & MC-460, ML-768, FR-976, HQC-192 (8) & 13--541 & 0.01--1.49 & 4.22--4.35 \\
Level 5 & MC-8192, ML-1024, HQC-256 (9) & 10--1032 & 0.06--2.39 & 4.28--4.35 \\
\bottomrule
\end{tabular}
\endgroup
\end{table}

Three key observations emerge. First, lattice-based ML-KEM suites dominate handshake speed across all security levels, completing in 9--20~ms while maintaining $\geq$99\% packet delivery, explaining why the expert policy favors them when delay alarms fire. Second, code-based suites inherit multi-hundred millisecond handshakes even without DDoS workloads; the 1.09~s handshake for \texttt{MC-348-AG-SP-128f} illustrates the scheduler's need to plan around reconnect penalties despite acceptable throughput and power. Third, power draw is effectively flat across levels ($\leq$0.27~W spread), confirming that battery-driven guard actions stem from sustained CPU residency rather than cipher choice itself.

\subsection{The Dominance of Latency over Power}

Our central finding is that while PQC handshake latency varies dramatically, the impact on steady-state power consumption is negligible. Table~\ref{tab:rep_suites} summarizes three representative suites, showing a >100$\times$ difference in handshake latency but a <2\% variance in average power draw.

\begin{table}[ht]
\centering
\caption{Representative Suites: Latency vs. Power Summary}
\label{tab:rep_suites}
\begingroup
\small
\setlength{\tabcolsep}{3pt}
\begin{tabular}{@{}p{0.30\linewidth}rrr@{}}
\toprule
\textbf{Suite} & \textbf{HS (ms)} & \textbf{Prim (ms)} & \textbf{Pwr (W)} \\
\midrule
ML-512-CP-FA-512 & 9.27 & 0.37 & 4.28 \\
ML-768-AG-MD-65 & 35.50 & 1.99 & 4.31 \\
MC-8192-AG-SP-256f & 913.08 & 555.85 & 4.33 \\
\bottomrule
\end{tabular}
\endgroup
\end{table}

This proves that the primary operational cost of PQC is in **responsiveness and link recovery time**, not battery endurance, when the cryptographic stack runs in isolation. As shown in Table~\ref{tab:powerbands}, the total energy consumed during a 45-second mission is nearly identical regardless of the KEM family used. When the transformer load is enabled, average power rises to 4.58--4.67~W, confirming that the additional draw is attributable to the co-located detection workload rather than the choice of KEM. This validates our scheduler's focus on latency and link quality as primary decision metrics.

\begin{table}[ht]
\centering
\caption{Power Bands by KEM Family (Range at 8~Mb/s, 45~s)}
\label{tab:powerbands}
\begingroup
\small
\setlength{\tabcolsep}{4pt}
\begin{tabular}{@{}lrrr@{}}
\toprule
\textbf{KEM Family} & \textbf{Min (W)} & \textbf{Max (W)} & \textbf{Energy (J)} \\
\midrule
ML-KEM & 4.279 & 4.343 & 192.5--195.5 \\
McEliece & 4.270 & 4.354 & 192.2--195.9 \\
HQC & 4.243 & 4.372 & 191.0--196.7 \\
FrodoKEM & 4.272 & 4.345 & 192.2--195.5 \\
\bottomrule
\end{tabular}
\endgroup
\end{table}

\subsection{Per-Family PQC Performance Analysis}

\subsubsection{ML-KEM (Kyber) Suites}
ML-KEM provides the best performance, with handshake times under 40~ms. The data in Table~\ref{tab:mlkem_results} reveals that the signature choice is the dominant cost factor; switching from Falcon-512 to SPHINCS$^+$-256f increases latency by over 12$\times$. The scheduler selects these suites for high-priority actions like link recovery.

\begin{table}[ht]
\centering
\caption{ML-KEM Results (Selected Suites)}
\label{tab:mlkem_results}
\begingroup
\footnotesize
\setlength{\tabcolsep}{2pt}
\begin{tabular}{@{}p{0.26\linewidth}rrrrr@{}}
\toprule
\textbf{Suite} & \textbf{HS} & \textbf{Prim} & \textbf{RTT} & \textbf{Loss} & \textbf{Pwr} \\
 & \textbf{(ms)} & \textbf{(ms)} & \textbf{(ms)} & \textbf{(\%)} & \textbf{(W)} \\
\midrule
ML-512-AG-FA-512 & 13.32 & 4.60 & 15.14 & 0.25 & 4.34 \\
ML-512-CP-MD-44 & 8.09 & 0.38 & 12.89 & 0.09 & 4.30 \\
ML-768-AG-MD-65 & 35.50 & 1.99 & 12.17 & 0.03 & 4.31 \\
ML-1024-CP-FA-1024 & 9.67 & 1.35 & 18.54 & 0.12 & 4.28 \\
ML-1024-AG-SP-256f & 165.44 & 133.00 & 23.36 & 1.86 & 4.29 \\
\bottomrule
\end{tabular}
\endgroup
\end{table}

\subsubsection{Classic-McEliece \& HQC Suites}
Code-based schemes like McEliece and HQC (Tables~\ref{tab:mceliece_results} \&~\ref{tab:hqc_results}) trade high initial latency for strong security assumptions. The 913~ms handshake of MC-8192-AG-SP-256f is unsuitable for time-critical maneuvers but is viable for non-urgent rekeying during stationary flight phases, a decision our scheduler makes explicitly. HQC-256 exhibited higher packet loss, making it a less desirable choice under poor link conditions.

\subsection{Network Throughput and Link Quality Analysis}

The viability of a UAV's communication link is primarily indicated by network throughput, goodput, and packet loss rate. These metrics directly reflect the system's ability to transmit and receive critical data reliably. Baseline trials delivered 7.24--7.93~Mb/s (about 91--99\% of target), the XGBoost posture held 7.66--7.94~Mb/s (96--99\%), and the transformer posture sustained 7.37--7.81~Mb/s (92--98\%), underscoring the scheduler's ability to prioritize data movement even as background load increases.

However, the introduction of the heavyweight TST model introduced a notable trade-off in link quality. While the Baseline and Lightweight scenarios generally exhibited high packet delivery ratios, the Heavyweight scenario saw a marked increase in packet loss. For instance, with \texttt{MC-348-AG-SP-128f}, packet loss increased from 0.040\% in baseline to a significant 6.447\% under heavy TST load. Even the Lightweight (XGBoost) model introduced measurable stress, with outliers such as \texttt{HQC-256-AG-MD-87} and \texttt{FR-640A-CP-MD-44} experiencing 3.226\% and 2.459\% loss, respectively. This suggests that intermittent processing spikes of the XGBoost model, when combined with certain cryptographic workloads, can begin to saturate system resources. As CPU resources become contended, the scheduler sheds load by dropping packets---a form of graceful degradation to preserve the primary data-link objective.

\subsection{Performance under Lightweight Computational Load (XGBoost)}

Introducing the lightweight XGBoost-based DDoS detection model creates mild resource contention. This workload modestly impacts performance, revealing how different algorithms respond to background CPU activity. Table~\ref{tab:lightweight_comparison} compares selected suites.

\begin{table}[ht]
\centering
\caption{Lightweight Load Performance (XGBoost DDoS Detection)}
\label{tab:lightweight_comparison}
\begingroup
\footnotesize
\setlength{\tabcolsep}{2pt}
\begin{tabular}{@{}p{0.26\linewidth}rrrr@{}}
\toprule
\textbf{Suite} & \textbf{HS} & \textbf{RTT} & \textbf{Loss} & \textbf{Pwr} \\
 & \textbf{(ms)} & \textbf{(ms)} & \textbf{(\%)} & \textbf{(W)} \\
\midrule
ML-512-AG-MD-44 & 14.29 & 17.35 & 0.53 & 4.321 \\
ML-512-AG-FA-512 & 13.32 & 15.14 & 0.25 & 4.343 \\
ML-768-AG-MD-65 & 35.50 & 12.17 & 0.03 & 4.307 \\
ML-1024-AG-MD-87 & 15.67 & 21.73 & 2.02 & 4.287 \\
ML-1024-AG-FA-1024 & 10.96 & 25.05 & 1.02 & 4.314 \\
HQC-128-AG-FA-512 & 65.50 & 12.54 & 0.07 & 4.372 \\
HQC-256-AG-MD-87 & 345.28 & 61.16 & 3.23 & 4.243 \\
MC-348-AG-SP-128f & 253.70 & 12.57 & 0.10 & 4.354 \\
MC-8192-AG-SP-256f & 913.08 & 14.27 & 0.56 & 4.329 \\
\bottomrule
\end{tabular}
\endgroup
\end{table}

Under lightweight load, most suites exhibit moderate increases in network latency and packet loss. Interestingly, several suites showed \textit{decreased} handshake latency and RTT. For instance, \texttt{HQC-128-AG-FA-512} handshake time decreased from 101.06~ms to 65.50~ms, and RTT dropped from 17.52~ms to 12.54~ms. This counter-intuitive result may be attributable to changes in process scheduling, CPU cache state, or other system-level effects when moving from idle to consistent low-level background processing. The most intensive suite, \texttt{MC-348-AG-SP-128f}, saw handshake time improve dramatically from 1090.40~ms to 253.70~ms, suggesting baseline performance may have been anomalous or subject to cold-start penalties mitigated by continuous workload.

\subsection{Performance under Heavyweight Computational Load (Transformer)}

The heavyweight Transformer-based DDoS model places the system under significant computational stress, exposing severe performance degradation and highlighting breaking points for several cryptographic suites. Table~\ref{tab:heavyweight_comparison} shows results under heavy load.

\begin{table}[ht]
\centering
\caption{Heavyweight Load Performance (Transformer DDoS Detection)}
\label{tab:heavyweight_comparison}
\begingroup
\footnotesize
\setlength{\tabcolsep}{2pt}
\begin{tabular}{@{}p{0.26\linewidth}rrrr@{}}
\toprule
\textbf{Suite} & \textbf{HS} & \textbf{RTT} & \textbf{Loss} & \textbf{Pwr} \\
 & \textbf{(ms)} & \textbf{(ms)} & \textbf{(\%)} & \textbf{(W)} \\
\midrule
ML-512-AG-MD-44 & 8.14 & 27.92 & 2.36 & 4.620 \\
ML-512-AG-FA-512 & 5.33 & 38.27 & 3.33 & 4.610 \\
ML-768-AG-MD-65 & 22.74 & 34.21 & 3.07 & 4.612 \\
ML-1024-AG-MD-87 & 12.45 & 38.07 & 4.67 & 4.672 \\
ML-1024-AG-FA-1024 & 15.67 & 35.07 & 6.41 & 4.671 \\
HQC-128-AG-FA-512 & 73.81 & 31.27 & 2.11 & 4.695 \\
HQC-256-AG-MD-87 & 322.94 & 38.08 & 5.56 & 4.677 \\
MC-348-AG-SP-128f & 837.13 & 110.38 & 6.45 & 4.585 \\
MC-8192-AG-SP-256f & 1637.19 & 45.14 & 3.30 & 4.670 \\
\bottomrule
\end{tabular}
\endgroup
\end{table}

The heavyweight load caused dramatic decline in network quality for all suites. Packet loss rates increased substantially, often exceeding 3--6\%. The \texttt{MC-348-AG-SP-128f} suite experienced catastrophic RTT increase from 13.56~ms (baseline) to 110.38~ms, coupled with packet loss increase from 0.04\% to 6.45\%, rendering the C2 link unreliable. Its handshake time regressed to 837.13~ms. The larger \texttt{MC-8192-AG-SP-256f} saw handshake time balloon to 1637.19~ms.

\textbf{Tail Latency Analysis.} For UAV command and control (C2) applications, low and predictable latency is critical for mission success. An analysis of tail latency is particularly revealing; the 95th percentile (p95) RTT for \texttt{MC-348-AG-SP-128f} escalated from 28.647~ms in baseline to 493.777~ms in heavyweight tests. This dramatic increase in p95 and maximum RTT values is evidence of extreme queuing delays and CPU contention. While average user experience degrades, the worst-case experience collapses, creating latency spikes that could jeopardize real-time control. This behavior is a direct consequence of the scheduler managing severe resource contention between network processing, cryptographic operations, and the demanding TST model.

In contrast, lattice-based suites demonstrated greater resilience. While \texttt{ML-512-AG-FA-512} still suffered increased RTT (18.09~ms to 38.27~ms) and packet loss (1.31\% to 3.33\%), handshake latencies remained low, indicating session re-establishment would be fast even under duress.

\subsection{Cross-Scenario Comparative Analysis}

A holistic cross-scenario analysis reveals that computational load impact is not uniform across PQC algorithm classes. Performance degradation is most pronounced for suites whose primary bottleneck is CPU-intensive computation.

\textbf{Code-based KEMs} (Classic-McEliece) suffer the most significant handshake latency increases. \texttt{MC-460-AG-MD-65} handshake time increased by 98\% from baseline (293.67~ms) to heavyweight (580.72~ms). In contrast, \texttt{ML-768-AG-MD-65} saw only 17\% increase (19.40~ms to 22.74~ms) under the same conditions.

\textbf{Signature schemes with high computational costs} are disproportionately affected by CPU contention. Any suite using SPHINCS$^+$ sees handshake latency dominated by signing operations. For \texttt{ML-1024-AG-SP-256f}, handshake time increased 35\% from baseline to heavyweight load, almost entirely attributable to the signature signing primitive competing for CPU cycles with the Transformer model. Conversely, schemes with extremely fast signing (Falcon, ML-DSA) contribute minimally to handshake latency even under heavy load. This demonstrates that for CPU-constrained platforms, signature scheme performance can become the single greatest limiting factor to C2 resilience, making fast-signing algorithms like Falcon and ML-DSA critically important design choices.

\subsection{Power and Energy Consumption Analysis}

Energy efficiency is critical for battery-powered UAVs. Table~\ref{tab:power_cross_scenario} analyzes average power consumption across the three scenarios, quantifying the energy cost of both cryptographic operations and ancillary on-board analytics.

\begin{table}[ht]
\centering
\caption{Power Consumption Across Three Scenarios}
\label{tab:power_cross_scenario}
\begingroup
\footnotesize
\setlength{\tabcolsep}{2pt}
\begin{tabular}{@{}p{0.26\linewidth}rrr@{}}
\toprule
\textbf{Suite} & \textbf{Base} & \textbf{Light} & \textbf{Heavy} \\
 & \textbf{(W)} & \textbf{(W)} & \textbf{(W)} \\
\midrule
ML-512-AG-MD-44 & 4.202 & 4.321 & 4.620 \\
ML-512-AG-FA-512 & 4.215 & 4.343 & 4.610 \\
ML-768-AG-MD-65 & 4.217 & 4.307 & 4.612 \\
ML-1024-AG-MD-87 & 4.352 & 4.287 & 4.672 \\
ML-1024-AG-FA-1024 & 4.344 & 4.314 & 4.671 \\
HQC-128-AG-FA-512 & 4.114 & 4.372 & 4.695 \\
HQC-256-AG-MD-87 & 4.316 & 4.243 & 4.677 \\
MC-348-AG-SP-128f & 4.234 & 4.354 & 4.585 \\
MC-8192-AG-SP-256f & 4.350 & 4.329 & 4.670 \\
\bottomrule
\end{tabular}
\endgroup
\end{table}

Under baseline conditions, most suites consumed 4.1--4.3~W with minimal variation. The lightweight XGBoost model resulted in consistent but small increase to 4.2--4.4~W. However, the heavyweight Transformer model imposed a significant energy penalty, uniformly increasing consumption to 4.5--4.7~W. This near-uniform 0.4W penalty, \textit{regardless of PQC suite}, implies the dominant energy cost under heavy load is the analytic model itself, not the cryptography. This reinforces the adaptive scheduler strategy: by selecting a more computationally efficient PQC suite, it can free up CPU cycles for the DDoS model, potentially allowing it to complete faster and return the CPU to lower power state, optimizing overall mission endurance.

\subsection{System Resource Utilization}

On a resource-constrained UAV platform, meticulous monitoring of CPU and memory utilization is essential to ensure system stability. The Resident Set Size (RSS) provides a measure of the memory footprint, while maximum CPU utilization indicates peak computational load. The experimental data reveals a clear and consistent trend of increasing resource consumption as the security posture escalates.

In the baseline scenario, memory footprint (RSS) was generally contained within 270--280~MiB. Activating the lightweight XGBoost model increased this to a range of 600--615~MiB. The heavyweight TST model further increased the memory footprint to 742--780~MiB across test suites. For \texttt{MC-348-AG-SP-128f}, RSS memory grew from 268.4~MiB (baseline) to 605.2~MiB (XGBoost) and finally to 749.0~MiB (Transformer).

A similar trend was observed in CPU utilization. While baseline and lightweight modes saw peak CPU usage in the 70--85\% range, the heavyweight TST scenario consistently pushed the CPU toward saturation, with maximum utilization frequently exceeding 90\%. Table~\ref{tab:resource_utilization} summarizes these trends.

\begin{table}[ht]
\centering
\caption{System Resource Utilization by Scenario}
\label{tab:resource_utilization}
\begingroup
\small
\setlength{\tabcolsep}{3pt}
\begin{tabular}{@{}lrr@{}}
\toprule
\textbf{DDoS Policy} & \textbf{Avg CPU (\%)} & \textbf{Avg RSS (MiB)} \\
\midrule
Baseline & 76.8 & 272.3 \\
Lightweight (XGBoost) & 78.6 & 606.4 \\
Heavyweight (TST) & 90.6 & 760.3 \\
\bottomrule
\end{tabular}
\endgroup
\end{table}

This analysis confirms that the heavyweight TST model imposes a significant and consistent resource overhead, which is the primary driver for the performance trade-offs observed in other key metrics like latency and packet loss.

\begin{table}[ht]
\centering
\caption{Classic-McEliece Results (Selected Suites)}
\label{tab:mceliece_results}
\begingroup
\footnotesize
\setlength{\tabcolsep}{2pt}
\begin{tabular}{@{}p{0.26\linewidth}rrrrr@{}}
\toprule
\textbf{Suite} & \textbf{HS} & \textbf{Prim} & \textbf{RTT} & \textbf{Loss} & \textbf{Pwr} \\
 & \textbf{(ms)} & \textbf{(ms)} & \textbf{(ms)} & \textbf{(\%)} & \textbf{(W)} \\
\midrule
MC-348-AG-SP-128f & 253.70 & 174.57 & 12.57 & 0.10 & 4.35 \\
MC-460-AG-MD-65 & 641.09 & 513.93 & 16.15 & 0.28 & 4.30 \\
MC-8192-AG-SP-256f & 913.08 & 555.85 & 14.27 & 0.56 & 4.33 \\
\bottomrule
\end{tabular}
\endgroup
\end{table}

\begin{table}[ht]
\centering
\caption{HQC Results (Selected Suites)}
\label{tab:hqc_results}
\begingroup
\footnotesize
\setlength{\tabcolsep}{2pt}
\begin{tabular}{@{}p{0.26\linewidth}rrrrr@{}}
\toprule
\textbf{Suite} & \textbf{HS} & \textbf{Prim} & \textbf{RTT} & \textbf{Loss} & \textbf{Pwr} \\
 & \textbf{(ms)} & \textbf{(ms)} & \textbf{(ms)} & \textbf{(\%)} & \textbf{(W)} \\
\midrule
HQC-128-CP-FA-512 & 58.07 & 6.69 & 11.40 & 0.18 & 4.33 \\
HQC-192-CP-MD-65 & 174.71 & 28.44 & 15.27 & 0.04 & 4.26 \\
HQC-256-AG-MD-87 & 345.28 & 61.32 & 61.15 & 3.23 & 4.24 \\
\bottomrule
\end{tabular}
\endgroup
\end{table}

\subsection{Expert Scheduler Evaluation}

The scheduler's effectiveness lies in its ability to pivot based on system state. Under simulated DDoS stress, the XGBoost detector correctly identified congestion and triggered a rekey to the lightweight suite \texttt{ML-1024-CP-MD-87} (HS~10.79~ms). That suite sustained 0.11\% loss in the XGBoost posture and 1.61\% under the transformer load, compared with 6.45\% loss when the system remained on \texttt{cs-classicmceliece348864-aesgcm-sphincs128fsha2} without adaptation. Table~\ref{tab:policy_power} compares the scheduler's best-case (low-power) and worst-case (high-performance) policy choices, showing that adaptation can trim $\sim$3\% of power and reduce peak CPU load by nearly 10 percentage points, directly improving energy-per-mission.

\begin{table}[ht]
\centering
\caption{Scheduler Lens: Best vs. Worst Policy Power (45~s)}
\label{tab:policy_power}
\begingroup
\small
\setlength{\tabcolsep}{4pt}
\begin{tabular}{@{}lrr@{}}
\toprule
\textbf{Metric} & \textbf{Worst} & \textbf{Best} \\
\midrule
Avg. Power (W) & 4.372 & 4.243 \\
Total Energy (J) & 196.74 & 190.95 \\
Max CPU (\%) & 73.7 & 64.9 \\
\bottomrule
\end{tabular}
\endgroup
\end{table}


\subsection{Quantitative Performance Trends}

This subsection synthesizes key performance trends identified from the aggregated experimental data, providing quantitative insights into the scheduler's behavior.

\textbf{Trend 1: Throughput Stability vs. Link Quality Degradation.} Average throughput remained remarkably stable across all three defense modes: 7.476~Mb/s (Baseline), 7.886~Mb/s (Lightweight), 7.632~Mb/s (Heavyweight). This highlights the scheduler's success in prioritizing data transmission. However, this stability comes at the cost of link quality when the heavyweight TST model is active. Average packet loss increased from 0.655\% (Baseline) to 0.697\% (Lightweight) and 3.637\% (Heavyweight)---a more than five-fold increase. This indicates that as CPU saturation occurs, the scheduler implements graceful degradation, sacrificing packet reliability to maintain consistent data rate and prevent total communication collapse.

\textbf{Trend 2: Energy Homogeneity Under Heavy Load.} Under the Heavyweight TST model, average power consumption across ML-KEM (4.632~W), Classic-McEliece (4.634~W), and HQC (4.636~W) families is functionally identical. While individual cryptographic suites have vastly different computational costs (e.g., 0.385~ms for \texttt{ML-512-CP-MD-44} vs. 555.850~ms for \texttt{MC-8192-AG-SP-256f}), the constant high-utilization workload of the TST model dominates the system's power signature. The DDoS model's overhead becomes the primary driver of power consumption, raising the entire system's power floor and masking the subtle performance differences between cryptographic algorithms.

\textbf{Trend 3: Compounding Effect of Cryptographic Intensity on Latency.} Suites with higher intrinsic cryptographic processing times experience more dramatic increases in network RTT under heavy load. \texttt{ML-768-AG-MD-65}, with low primitive latency (1.993~ms), sees RTT increase by 2.8$\times$ from 12.170~ms to 34.208~ms. In contrast, \texttt{HQC-192-AG-MD-65}, with much higher primitive latency (145.115~ms), experiences 3.5$\times$ increase in RTT from 26.206~ms to 92.107~ms. A high-cost primitive occupies the CPU for a longer continuous duration. When this coincides with the TST model's processing window, it creates a prolonged ``CPU unavailable'' state, causing network packet queues to build up significantly and leading to dramatic RTT spikes.

\textbf{Trend 4: Stepwise Resource Escalation.} Transitioning from Baseline to Lightweight policy results in marginal CPU increase (76.8\% to 78.6\%) but more than doubles memory footprint (272.3~MiB to 606.4~MiB). The transition to Heavyweight policy introduces substantial burden, pushing average peak CPU above 90\% and increasing memory by an additional 150~MiB to 760.3~MiB. This confirms that the heavyweight TST model imposes significant and consistent resource overhead, which is the primary driver for observed performance trade-offs.

\section{Discussion and Future Work}

\subsection{Discussion}

\textbf{Agility as a primary design axis.} Static ``one-size-fits-all'' cryptographic policies fail to accommodate the dynamic constraints of UAV missions. The results demonstrate that runtime agility---the ability to pivot between PQC suites based on live telemetry---provides substantial benefits in reliability, responsiveness, and endurance. This capability transforms cryptography from a static configuration parameter into a controllable runtime resource.

\textbf{Cost asymmetry of PQC components.} Empirical evidence indicates that digital signatures dominate total handshake cost, whereas AEAD encryption contributes minimally to overall delay or power draw. Schemes such as Falcon and ML-DSA provide excellent performance-to-security ratios, whereas SPHINCS$^+$ remains suitable for pre-flight authentication rather than frequent rekeys.

\textbf{Reliability as the limiting factor.} Loss and RTT spikes, particularly in HQC-256 and McEliece configurations under heavy load, highlight that network reliability rather than raw power limits mission continuity. This validates the scheduler's design choice to incorporate heartbeat and packet-loss metrics into policy selection logic.

\textbf{Static policies are untenable.} The cross-scenario analysis establishes that static cryptographic policies cannot maintain acceptable performance under realistic computational loads. Adaptive security is not optional but a fundamental prerequisite for mission-critical UAV systems in the post-quantum era.

\textbf{Mission planning implications.} Operational deployments should consider the following guidance based on empirical findings: ML-KEM-768 (Level~3) emerges as the operational sweet spot, delivering reliable throughput even under heavyweight computational loads while maintaining handshakes below 40~ms. Code-based suites (Classic-McEliece, HQC) provide conservative security margins but incur the steepest loss and reconnect penalties, making them better suited for high-trust, low-interruption missions where link stability is guaranteed. The observed loss shedding (up to 6.4\% for McEliece under heavy load) is deliberate: the scheduler prioritizes maintaining control-link availability at the expense of non-critical packets. Operators should monitor telemetry logs for sustained packet loss exceeding 3\% as a trigger to downshift mission tempo or disable heavyweight analytics pipelines. The guard pipeline's thermal and battery constraints provide natural safety bounds, but mission planners must account for the fact that heavy DDoS detection workloads uniformly raise power consumption by approximately 0.4~W regardless of cryptographic suite choice.

\textbf{Scheduler behavior patterns.} Field telemetry reveals three recurring guard reactions that validate the adaptive architecture: (i)~\textit{thermal-elevated} mode swaps high-power, long-handshake suites (e.g., \texttt{MC-460-AG-MD-65}) for cheaper ML-KEM or HQC options to prevent thermal runaway; (ii)~\textit{battery-low} mode favors suites with lowest measured average power (\texttt{HQC-256-AG-MD-87} at 4.243~W), extending endurance at the cost of slightly elevated loss; and (iii)~\textit{heartbeat-missing/DDoS} mode triggers rapid pivots to fast-handshake suites (\texttt{ML-1024-CP-MD-87} with 10.8~ms handshake) to minimize link recovery time during attacks. These patterns demonstrate the scheduler's ability to balance competing constraints in real time.

\subsection{Future Work}

\textbf{Reinforcement-Learning-Driven Scheduling.} Future iterations will replace the static lookup policy with a reinforcement-learning (RL) agent~\cite{watkins1992qlearning,hortelano_rl_edge_survey_2023} trained on the same telemetry space. The RL state vector will include CPU utilization, battery state-of-charge, network latency, packet loss, and rekey frequency. Actions will correspond to selecting PQC suites and triggering rekey events. The reward function will be defined as $R = (w_{sec} \cdot S) - (w_{lat} \cdot L) - (w_{pwr} \cdot P)$, where $S$ is a security score, $L$ is latency, $P$ is power, and $w_i$ are mission-phase-dependent weights.

\textbf{Mission-Phase-Aware Policy Bands.} Planned extensions will integrate flight-phase information (e.g., takeoff, hover, transit, landing) into the scheduler's context. High-security suites can be activated during stationary or low-dynamics phases, while lightweight configurations preserve energy during high-load maneuvers.

\textbf{Hierarchical Anomaly-Driven Adaptation.} Future versions will introduce a hierarchical anomaly classifier that distinguishes volumetric, jitter-based, and protocol-layer attacks. Each class will map to a corresponding scheduler response, enabling multi-layer resilience without external intervention.

\textbf{Multi-Agent Coordination.} In swarm or multi-UAV scenarios, distributed schedulers could coordinate suite selection to maintain group secrecy while optimizing total fleet power. Such coordination will require lightweight consensus protocols compatible with PQC primitives.

\textbf{Extended Benchmarking and Public Dataset.} Upcoming work will expand evaluation to include additional security levels, future lightweight ciphers (Ascon), and various network topologies and wireless channels. A public dataset containing synchronized power, latency, and telemetry traces will be released to facilitate reproducibility and comparative research.

\textbf{Instrumentation Refinements.} Baseline runs occasionally exhibited sporadic artifacts (e.g., a 652~ms handshake for \texttt{FR-640A-CP-MD-44} that contradicts the suite's typical sub-60~ms performance), likely caused by idle-core sleep states or thermal throttling transitions. Future campaigns should enforce fixed CPU frequency pinning and disable power-saving modes to eliminate this measurement bias. Additionally, only the deterministic expert lookup policy was evaluated in this study; extending the same telemetry framework to the dormant reinforcement-learning (\texttt{strategies/rl.py}) and hybrid (\texttt{strategies/hybrid.py}) strategies will clarify whether learned policies can outperform rule-based selections without violating guard constraints.

\section{Conclusion}

This paper addressed the critical challenge of implementing high-performance, quantum-resistant cryptography on resource-constrained UAV platforms. We presented a complete C2 stack designed for resilience, featuring a modular architecture and an adaptive scheduling policy engine informed by real-time telemetry. Our comprehensive evaluation systematically quantified the performance of a wide range of PQC suites under varying levels of computational stress, simulating the presence of on-board DDoS detection models.

The key findings from our three-scenario evaluation underscore the significant and non-uniform impact of PQC on system performance. We demonstrated clear trade-offs between NIST security level, network performance (latency and packet loss), and power consumption. These trade-offs are dramatically exacerbated by ancillary computational loads, where suites with high CPU requirements (Classic-McEliece, SPHINCS$^+$) suffer severe degradation in C2 link reliability. Conversely, lattice-based schemes (ML-KEM, ML-DSA) proved far more resilient. While handshake latency varies by over 100$\times$ (8~ms to 1637~ms under heavy load), steady-state power consumption remains remarkably stable (4.1--4.7~W across all scenarios), proving that the primary operational cost of PQC is in transient latency and link reliability, not sustained energy drain.

Our analysis quantified the significant energy cost associated with advanced on-board analytics, reinforcing the need for intelligent resource management. These results validate the central thesis of our work: \textbf{an adaptive security architecture is a necessary mechanism for maintaining operational resilience in secure UAV systems}. Static cryptographic policies are operationally untenable under realistic computational loads. By enabling graceful degradation of cryptographic policies, our proposed system allows a UAV to balance its security posture against mission-critical performance requirements in real-time.

The findings of this study provide a foundational data set for the design of future secure UAVs and inform the development of sophisticated scheduling policies. Future research will focus on the implementation of machine learning-based scheduling strategies that can learn optimal policies from operational data, further enhancing the autonomy and resilience of next-generation unmanned systems. The proposed architecture transforms PQC from a static configuration into a dynamic, mission-aware subsystem, bridging the gap between post-quantum cryptography, cyber-resilience, and intelligent resource management in mission-critical UAV deployments.


%%
%% Print the bibliography
%%
\printbibliography

\end{document}
\endinput
%%
%% End of file `sample-sigconf-biblatex.tex'.